\documentclass[10pt, a4paper]{article}

\title{Work Process}
\author{Group 15}
\date{January 25, 2024}

\begin{document}
\maketitle
\tableofcontents

\section{Overall Work Distribution}
\subsection{Martin}
\begin{itemize}
	\item Took major parts in designing all versions of the project's infrastructure.
	\item Documentation writing (Project Plan, Design Document).
	\item Terraform scripting.
	\item Programming (Flask API, Lambda functions: createMSK, Login).
	\item Managing and Configuring API Gateways.
	\item Authentication flow (Cognito, IAM).
	\item Route53 connection to Cognito.
	\item MSK Authentication and testing.

\end{itemize}

\subsection{Evgeni}
\begin{itemize}
	\item Took major parts in designing all versions of the project's infrastructure.
	\item Documentation writing (Project Plan, Design Document).
	\item Terraform scripting.
	\item Programming (Flask API, Lambda functions: deleteMSK, CloudWatch dashboard).
	\item Managing and Configuring API Gateways.
	\item Monitoring (CloudWatch).
	\item Route53 connection to API.
\end{itemize}

\subsection{Nikola}
\begin{itemize}
	\item Documentation writing (Project plan, Design document)
	\item Ansible scripting
	\item Route 53 initial setup
	\item Route 53 connection to the API gateway
	\item Duplication of resources in region Ireland
\end{itemize}


\section{Mark justification}
\subsection{Martin}
Now, I don't like giving grades to myself since I am quite self-critical. But this time, I know that I gave it my all. I went through all the topics covered in the course material. I proved my abilities in either the Case Study project, in the assignments, or both. Not only did I learn so many things, but I delved deep into them, deep enough so I could find AWS documentation inconsistencies and loopholes. I took a risk to create a project that isn't so common instead of going the easy path of bland overused concepts. And I do not regret it since I learned so much. Technically, some things don't work but we couldn't foresee such outcomes since we implemented everything as AWS suggests. So, in the end, I am proud of the results and evaluate myself at least at a Proficient level.

\subsection{Evgeni}
During ths semester I think I managed to learn a lot. I spent a lot of time researching and understanding how the different AWS tools work,but I think it wsa worth it
since I reached the point of finding tickets sent from developers to AWS who found the same problems as I do. Sure our project wasn't the most convenient one, but we 
managed to do it and I managed to learn a lot of things that I wasn't going to if it wasn't the project. I did my best in this semester and I think that a grade of Proficient or above fits my performance. 

\subsection{Nikola}
This semester wasn't easy on me by any means. 
    However, I'm satisfied with my result, because due to the ambitious scope of our project, 
    I was able to learn a lot more than wat was even included during the classes. And I'm satisfied with that.
    This entire semester was a journey filled with lots of spontaneous troubleshooting and adjustments, but that's what makes alll the more beneficial, since the best way to learn is by doing (or fixing what you've done).
    So as a conclusion I can say that I'm finishing this semester with a proficient grade.


\section{Individual reflections}
\subsection{Martin}
\subsubsection{Stong points}
\begin{itemize}
	\item Being responsible.
	\item Ensuring quality over quantity.
	\item Validating information sources.
	\item Learning services that enterprises use, although they are out of scope.
	\item Not relying on step-by-step tutorials and AI models.
\end{itemize}


\subsubsection{Weak points}
\begin{itemize}
	\item Relying too much on the documentation (AWS tends to sugarcoat their products) - I had some occurrences where I could have saved so much time by just reading a simple forum thread about a service I am about to use.
	\item At some point, I lost sight of the bigger picture because I was over-optimizing.
\end{itemize}

\subsubsection{Learning moments}

Throughout the semester, I had countless opportunities to hone my troubleshooting skills.

I also learned to not concentrate too much on the researching aspect of a project since documentation takes you so far. You will know if a service could be useful once you start utilizing it. And not just using it, but most importantly - breaking it.

\subsubsection{Improvements for the next project}
\begin{itemize}
    \item Benefit from some real-world opinions synthesized by experience before you start researching or implementing something. 
    \item Start small and simple and go from there. Don't throw yourself in a sea full of things you haven't even heard of.
    \item Don't forget the end goal - the project should have an overall completeness.
\end{itemize}


\subsection{Evgeni}
\subsubsection{Stong points}
\begin{itemize}
	\item Always present and took part in all parts of the project.
	\item Does his part of the work.
	\item Communicate with the teammates.
\end{itemize}


\subsubsection{Weak points}
\begin{itemize}
	\item Takes too much time to research (dives too deep).
\end{itemize}

\subsubsection{Learning moments}

The entire semester was a learning opportunity because I was constantly researching and troubleshooting Evaluation of spent effort: I did my tasks and took part as much as I could. I think I did everything I could do on the project and I can say that I spent a good amount of time working on it every week.


\subsubsection{Improvements for the next project}
\begin{itemize}
    \item Research in a more optimal way without spending too much time on the side tasks.
    \item Be more focused on the goal.
    \item Improve my communication with the stakeholders (reply to emails more often and check the mailbox).
\end{itemize}


\subsection{Nikola}
\subsubsection{Stong points}
\begin{itemize}
	\item Punctuality.
	\item Whatever work is handed to me, I'll do my best to complete it.
	\item Communication with teammates. 
\end{itemize}
\subsubsection{Weak points}
\begin{itemize}
	\item Work performance decreases if I find myself unable to solve an issue for longer than two or three hours.
\end{itemize}
\subsubsection{Learning moments}
Apart from the troubleshooting and adjustment tweaking that I had to do, I was also able to hone a skill that's 
        i didn't expect - by the end of the semester, when I felt that I'm starting to get agitated due to something not working as it should (see 3.3.3),
        I was able to step back, and give myself the opportunity to look at the situation with a clearer mind. This has actually helped me 
        find a way to improve something on many occasiions.
\subsubsection{Improvements for the next project}
\begin{itemize}
	\item Conducting research, that will yield sufficient results, in order to avoid spontaneous and unexpected complications.
        \item Try to assume more of an organisational role - so that I can also make more adequate propositions in the initial stages.
\end{itemize}

\end{document}

